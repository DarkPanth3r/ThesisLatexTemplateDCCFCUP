\chapter{Introdução}\label{chap:intro}

Algumas referências para se ver a ordenação~\cite{yaacoub2012}. Aqui outra ainda~\cite{etsitr102732}.

Deve-se ler Strunk~\cite{strunk2007elements}

\ac{HTTP} é um protocolo como diz em~\cite{test2000} e também na secção~\ref{chap:stat}.

  Temos aqui o Clausen~\cite{Clausen2003}. E para ver várias~\cite{yaacoub2012, etsitr102732, strunk2007elements}

Deve-se acrescentar os acrónimos no ficheiro \texttt{acros.tex} e ordená-los alfabeticamente nesse ficheiro.

\section{Referências}
Ver \url{https://www.liu.se/ikk/asv/kv/1.239281/HarvardGuidev3.0.1.pdf}

\section{Sugestões}
\begin{itemize}
  \item usar o \textasciitilde{} para ligar as referências, evita a possível mudança de página: ex.: como o Brandão refere\textasciitilde\textbackslash{}cite\{bran99\}
  \item em inglês não usar a \emph{short form}. Em textos formais deve-se manter a \emph{long form}. Incorreto: ``don't use''. \textbf{Correto:} ``do not use''
\end{itemize}
